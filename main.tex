%%%%%%%%%%%%%%%%%%%%%%%%%%%%%%%%%%%%%%%%%
% "ModernCV" CV and Cover Letter
% LaTeX Template
% Version 1.1 (9/12/12)
%
% This template has been downloaded from:
% http://www.LaTeXTemplates.com
%
% Original author:
% Xavier Danaux (xdanaux@gmail.com)
% 
% Updated by: Armin Hadzic
%
% License:
% CC BY-NC-SA 3.0 (http://creativecommons.org/licenses/by-nc-sa/3.0/)
%
% Important note:
% This template requires the moderncv.cls and .sty files to be in the same
% directory as this .tex file. These files provide the resume style and themes
% used for structuring the document.
%
%%%%%%%%%%%%%%%%%%%%%%%%%%%%%%%%%%%%%%%%%

%----------------------------------------------------------------------------------------
%   PACKAGES AND OTHER DOCUMENT CONFIGURATIONS
%----------------------------------------------------------------------------------------

\documentclass[12pt,a4paper,sans]{moderncv} % Font sizes: 10, 11, or 12; paper sizes: a4paper, letterpaper, a5paper, legalpaper, executivepaper or landscape; font families: sans or roman

\moderncvstyle{banking} % CV theme - options include: 'casual' (default), 'classic', 'oldstyle' and 'banking'
\moderncvcolor{black} % CV color - options include: 'blue' (default), 'orange', 'green', 'red', 'purple', 'grey' and 'black'

\usepackage[scale=0.88, 
			lmargin=0.95cm, 
			rmargin=0.95cm, 
			footnotesep=1.5cm
			]{geometry} % Reduce document margins

%bibliograph content
\usepackage[minnames=8,
			maxnames=10,
			backend=bibtex,
            style=authoryear,
            natbib=true, 
            style=numeric-comp,
            sorting=ydnt               
            ]{biblatex}
\usepackage[unicode,
			colorlinks=true,
			urlcolor=blue
			]{hyperref}    
\usepackage{booktabs}


% Publication subheadings
\defbibheading{journal}{Journal Publications}
\defbibheading{conference}{Publications}
\defbibheading{workshop}{Workshop Publications}
\defbibheading{abstracts}{Abstracts}

% Publication bib files
\addbibresource[label=conference]{conferences}
\addbibresource[label=journal]{journals}
\addbibresource[label=workshop]{workshops}
\addbibresource[label=under_review]{under_review}

%Title spacing
\makeatletter
\patchcmd{\makehead}{%search
    \ifthenelse{\equal{\@title}{}}{}{\titlestyle{~|~\@title}}\\%
    }{%replace
    \ifthenelse{\equal{\@title}{}}{}{\titlestyle{~|~\@title}}\\[0.5cm]%
  }{%success
  }{%failure
  }
\makeatother

%----------------------------------------------------------------------------------------
%   NAME AND CONTACT INFORMATION SECTION
%----------------------------------------------------------------------------------------

\firstname{Raymond} % Your first name
\familyname{Lesiyon} % Your last name

% All information in this block is optional, comment out any lines you don't need
%\title{Curriculum  LaTeX Error: Missing \begin{document}.Vitae}
\address{Aurora Colorado -80011}
%\mobile{(\#\#\#) \#\#\#-\#\#\#\#}
\email{raymondlesiyon@cuanschutz.edu}
% The first argument is the url for the clickable link, the second argument is the url displayed in the template - this allows special characters to be displayed such as the tilde in this example
\social[linkedin][https://www.linkedin.com/in/raymond-lesiyon-86a431146/]{raymond-lesiyon}
\social[github][https://github.com/rlesiyon]{rlesiyon}

%----------------------------------------------------------------------------------------


\begin{document}

\makecvtitle % Print the CV title
\vspace*{-1.5cm}

%----------------------------------------------------------------------------------------
%   RESEARCH INTERESTS SECTION
%----------------------------------------------------------------------------------------
\section{Research Interests}

\cvitem{Areas of interest}{Computational Biolgy, Natural Language Processing, Large Language Models, Protein Language Models, Knowledge Graphs, Network Analysis}

%----------------------------------------------------------------------------------------
%   EDUCATION SECTION
%----------------------------------------------------------------------------------------

\section{Education}
\cventry{2021-2023}{Master of Science in Computational Math Science and Engineering (CMSE)}{Michigan State University}{East Lansing, MI}{\textit{GPA - 3.94}}{} 
\begin{itemize}
    \item Coursework: Mathematical Foundation for Data Science, Numerical Linear Algebra, Data Mining, Statistical Genetics, Genomic Data Handling: Unix and Python, Genomics and Sequencing Analysis, RNA-Seq Data Analysis
\end{itemize}
\cventry{2017-2021}{B.Sc in Biosystems Engineering | Minor in CMSE}{Michigan State University}{East Lansing, MI}{\textit{GPA - 3.92}}{} 
\begin{itemize}
    \item Graduated Magna Cum Laude
    \item MasterCard Foundation Scholar Program Recipient
    \item Dean's Honors List : Fall 2021 - Spring 2021
    \item MasterCard Foundation Scholar Program Recipient, 2017
\end{itemize}

%----------------------------------------------------------------------------------------
%   Research experience
%----------------------------------------------------------------------------------------

\section{Research experience}

\cventry{Aurora, Colorado}{\textsc{University of Colorado, Anschutz}}{Informatic Professional}{2023-Present}{}
{
    \cvitem{Labs}{\textbf{JRaviLab}, \small{Department of Biomedical Informatics | CU Anschutz}, \textbf{Wale Lab}, \small{Department of Microbiology - MSU}}
    \cvitem{Mentors}{Dr. Janani Ravi, Dr. Nina Wale}
}
{
     \cvitem{Project 1}{Evolution of Bacterial Traits in Relation to its Host Association and Pathogenicity}
    \begin{itemize} 
        \item Enhanced data integration pipeline using tidyverse to merge 26 bacterial and archaea datasets, significantly improving data quality and accessibility. 
        \item Boosted database accuracy by integrating BacDive, and BugPhyzz, increasing data completeness from 20 to 50 and enabling superior research outcomes through expert data management. 
        \item Streamlined phylogenetic regression analysis with modular workflow, accelerating 200+ model iterations and enhancing the understanding of bacterial phenotypes and pathogenicity.
        \item Implemented advanced statistical analysis using AIC to evaluate and optimize phylogenetic regression, enhancing the accuracy and reliability of evolutionary biology research.
        \item Collaborated with 3 Principal Investigators to develop a robust bacterial traits analysis, leading to publications in preparation.
    \end{itemize}
}
{
    \cvitem{Project 2}{microgenomeR R data package development}
    \begin{itemize} 
        \item Adapted Austraits multiple datasets integration pipeline to effectively harmonize 26+ microbial datasets into a central source with consistent units and trait values. 
        \item Facilitated seamless loading of microbial traits data at strain and species levels using parquet and rda file format.
    \end{itemize}
}

\cventry{East Lansing}{\textsc{Michigan State University}}{Technical Aide}{Jun 2021- Aug 2021}{}
{
    \cvitem{Labs}{\textbf{Steibel Juan Lab}, \small{Department of Animal Science}, MSU}
    \cvitem{Mentors}{Dr. Steibel Juan}
    \begin{itemize}
        \item Leveraged Slurm workload manager for running LSTM deep learning algorithms on high-performance computing clusters, achieving optimized resource utilization and computational efficiency. 
        \item Fine-tuned m,hyper-parameters for enhanced LSTM deep learning model performance.
    \end{itemize}
}

\cventry{East Lansing}{\textsc{Fraunhofer USA Inc.}}{Biosensor Intern}{Jun 2019- Aug 2020}{}
{
    \cvitem{Mentors}{Dr. Suzanne Witt}
    \begin{itemize}
        \item Designed and implemented a data visualization dashboard with Tkinter and Pandas, streamlining the analysis of data from 37 biosensor fabrication experiments and enhancing decision-making.
        \item Functionalized antibody biosensor using N-hydroxysuccinimide (NHS) immobilization technique to boron-doped diamond surface for the purpose of detecting COVID-19 spike protein
        \item Involved in testing immobilized antibody biosensors using electrochemical impedance to ascertain the binding of COVID-19 spike protein.
    \end{itemize}
}

\cventry{East Lansing}{\textsc{Michigan State University}}{Undergraduate Research Assistant}{Jun 2019- Aug 2020}{}
{
    \cvitem{Labs}{\textbf{MIDI Lab}, \small{Department of Biomedical Engineering}, MSU}
    \cvitem{Mentors}{Dr. Adam Alessio}
    \begin{itemize}
        \item Performed single feature classification on 400 Ovarian torsion datasets using decision trees classifiers, random trees, and logistic regression through Sci-kit learn library in Python
        \item Employed ROC curve, accuracy, specificity, and sensitivity methodology to access the performance of ovarian data classification model
    \end{itemize}
}

%----------------------------------------------------------------------------------------
%   SKILL Section
%----------------------------------------------------------------------------------------
\section{SKILLS}

\cvitem{Programming Languages}{Python, R, Matlab, C++}
\cvitem{Computational tools}{Unix/Linux, Git, High-performance computing clusters}
\cvitem{Methodologies}{Regression, dimensional reduction - Principal Component Analysis(PCA), data analysis \& visualization - tidyvverse, ggplot, pandas}


%--------------------------------------------------------------------------------------
%   Publication
%--------------------------------------------------------------------------------------

\section{Publications}

\begin{enumerate}
    \item bacteria-traits in-preparation
    \item microgenomeR in-preparation
    \item Junjie Jan, Janice Siegford, Dirk Colbry, \textbf{Raymond Lesiyon}, Anna Bosgraaf, Chen Chen, Tomas Norton, Juan Steibel. Evaluation of Computer Vision for Detecting Agonistic Behaviour of pigs in a Single-SpaceFeeding Stall Through Blocked Cross-Validation Strategies. 10.2139/ssrn.4098711
    \item Suzanne T Witt, Alexis Rogien, Diana Weiner, James R Siegenthaler, \textbf{Raymond Lesiyon}, Noelle Kurien, Robert Rechenberg, Nina Baule, Aaron Hardy, Michael Becker. Boron doped diamond thin films for the electrochemical detection of SARS-CoV-2 S1 Protein. 10.1016/j.diamond.2021.108542
\end{enumerate}

%--------------------------------------------------------------------------------------
%   Presentation
%--------------------------------------------------------------------------------------

\section{Presentation}

\subsection{Research and Technical Talks} 
{
    \begin{itemize}
        \item {\cvitem{July 24}{\textbf{Bioc2024 International Conference}, microgenomeR an R data package for aggregating microbial phenotypic, \& genotypic trait. \\Val Andel Institute, Grand Rapids Michigan} }
    \end{itemize}
}

\subsection{Posters} 
{
    \begin{itemize}
        \item {\cvitem{Aug 24}{\textbf{CU Anschutz Department of Biomedical Informatics Retreat}, Are (bacterial) pathogens special?. \\Aurora, CO} }
        \item {\cvitem{June 24}{\textbf{Quantitative Cell \& Molecular Biology Symposium}, Are (bacterial) pathogens special?. \\Colorado State University, Fort Collins, CO} }
        \item {\cvitem{April 24}{\textbf{American Society of Microbiology Rocky Mountain Branch}, Are (bacterial) pathogens special?. \\University of Colorado Boulder, Boulder, CO} }
    \end{itemize}
}

%----------------------------------------------------------------------------------------
%   Teaching experience
%----------------------------------------------------------------------------------------

\section{Teaching experience}
{
    \cventry{East Lansing}{\textsc{Michigan State University}}{Graduate Teaching Assistant}{Aug 2021- Fall 2023}{Department of Computational Math Science and Eng.}
    {
        \cvitem{CMSE 802}{Methods in Computational Modeling}
        \cvitem{CMSE 202}{Computational Modelling and Data Analysis II}
    }
    
    \cventry{East Lansing}{\textsc{Michigan State University}}{Academic Tutor}{Fall 2019- Fall 2021}{College of Engineering}
    {
        \cvitem{}{Student tutor for calculus and physics}
    }
}

%----------------------------------------------------------------------------------------
%  Leadership
%----------------------------------------------------------------------------------------

\section{Service and Leadership}
{
    \cventry{Baringo, Kenya}{\textsc{Kokwa Island Primary School}}{Educational Support Initiative}{Aug 2023}{}
    {
        \cvitem{}{Donated uniforms to 11 students from orphaned or low-income families, improving access to education.}
    }
    
    \cventry{East Lansing}{\textsc{Michigan State University}}{Academic Core Lead Tutor}{Spring 2020- Fall 2021}{Department of Computational Math Science and Eng.}
    {
        \cvitem{}{Guided others tutors to ensure smooth running of the tutoring center}
        \cvitem{}{Planned and organized review section for STEM courses before exam}
    }
    
    \cventry{East Lansing}{\textsc{Michigan State University}}{International Orientation Leader }{Aug 2018}{Office of International Students}
    {
        \cvitem{}{Guided a group of 10 international students in their first week of enrolling at MSU.}
    }
    
    
    \cventry{Bomet, Kenya}{\textsc{Kapkures community}}{Maji Safi ni Uhai Initiative}{Oct 2019}{}
    {
        \cvitem{}{Co-founderd Maji Safi ni Uhai Initiative, and secured \$4000 from MCF program to drill water kapkures community in Bomet, Kenya.}
    }

    \cventry{Baringo, Kenya}{\textsc{Baringo South}}{Community Need Assessment Leader}{Mar 2017 - May 2017}{}
    {
        \cvitem{}{Lead a group of 10 students from Education and Social Empowerment Program(EaSEP) conduct community need assessment in Ilchamus community. }
    }

    \cventry{Baringo, Kenya}{\textsc{Kokwa Primary School}}{Volunteer Teacher}{Jan 2016 - June 2016}{}
    {
        \cvitem{}{Volunteered to teach STEM classes to grade eight students}
    }
    
    \cventry{Baringo, Kenya}{\textsc{Kokwa Island}}{Community Need Assessment Guide}{Jan 2016}{}
    {
        \cvitem{}{Helped in community need assesment in Kokwa Island with patternship with Friends of Kenya Schools and Wildlife (FKSW)}
    }
    
}

\end{document}
