%%%%%%%%%%%%%%%%%%%%%%%%%%%%%%%%%%%%%%%%%
% "ModernCV" CV and Cover Letter
% LaTeX Template
% Version 1.1 (9/12/12)
%
% This template has been downloaded from:
% http://www.LaTeXTemplates.com
%
% Original author:
% Xavier Danaux (xdanaux@gmail.com)
% 
% Updated by: Armin Hadzic
%
% License:
% CC BY-NC-SA 3.0 (http://creativecommons.org/licenses/by-nc-sa/3.0/)
%
% Important note:
% This template requires the moderncv.cls and .sty files to be in the same
% directory as this .tex file. These files provide the resume style and themes
% used for structuring the document.
%
%%%%%%%%%%%%%%%%%%%%%%%%%%%%%%%%%%%%%%%%%

%----------------------------------------------------------------------------------------
%   PACKAGES AND OTHER DOCUMENT CONFIGURATIONS
%----------------------------------------------------------------------------------------

\documentclass[12pt,a4paper,sans]{moderncv} % Font sizes: 10, 11, or 12; paper sizes: a4paper, letterpaper, a5paper, legalpaper, executivepaper or landscape; font families: sans or roman

\moderncvstyle{banking} % CV theme - options include: 'casual' (default), 'classic', 'oldstyle' and 'banking'
\moderncvcolor{black} % CV color - options include: 'blue' (default), 'orange', 'green', 'red', 'purple', 'grey' and 'black'

\usepackage[scale=0.88, 
			lmargin=0.95cm, 
			rmargin=0.95cm, 
			footnotesep=1.5cm
			]{geometry} % Reduce document margins

%bibliograph content
\usepackage[minnames=8,
			maxnames=10,
			backend=bibtex,
            style=authoryear,
            natbib=true, 
            style=numeric-comp,
            sorting=ydnt               
            ]{biblatex}
\usepackage[unicode,
			colorlinks=true,
			urlcolor=blue
			]{hyperref}    
\usepackage{booktabs}


% Publication subheadings
\defbibheading{journal}{Journal Publications}
\defbibheading{conference}{Publications}
\defbibheading{workshop}{Workshop Publications}
\defbibheading{abstracts}{Abstracts}

% Publication bib files
\addbibresource[label=conference]{conferences}
\addbibresource[label=journal]{journals}
\addbibresource[label=workshop]{workshops}
\addbibresource[label=under_review]{under_review}

%Title spacing
\makeatletter
\patchcmd{\makehead}{%search
    \ifthenelse{\equal{\@title}{}}{}{\titlestyle{~|~\@title}}\\%
    }{%replace
    \ifthenelse{\equal{\@title}{}}{}{\titlestyle{~|~\@title}}\\[0.5cm]%
  }{%success
  }{%failure
  }
\makeatother

%----------------------------------------------------------------------------------------
%   NAME AND CONTACT INFORMATION SECTION
%----------------------------------------------------------------------------------------

\firstname{Raymond} % Your first name
\familyname{Lesiyon} % Your last name

% All information in this block is optional, comment out any lines you don't need
%\title{Curriculum  LaTeX Error: Missing \begin{document}.Vitae}
\address{Aurora Colorado -80011}
%\mobile{(\#\#\#) \#\#\#-\#\#\#\#}
\email{raymondlesiyon@cuanschutz.edu}
% The first argument is the url for the clickable link, the second argument is the url displayed in the template - this allows special characters to be displayed such as the tilde in this example
\social[linkedin][https://www.linkedin.com/in/raymond-lesiyon-86a431146/]{raymond-lesiyon}
\social[github][https://github.com/rlesiyon]{rlesiyon}

%----------------------------------------------------------------------------------------


\begin{document}

\makecvtitle % Print the CV title
\vspace*{-1.5cm}

%----------------------------------------------------------------------------------------
%   RESEARCH INTERESTS SECTION
%----------------------------------------------------------------------------------------
\section{Research interests}

\cvitem{Areas}{Computational Biology,  Network Analysis, Knowledge Graphs, Natural Language Processing (NLP), Large Language Models (LLM), Protein Language Models}

%----------------------------------------------------------------------------------------
%   EDUCATION SECTION
%----------------------------------------------------------------------------------------

\section{Education}
\cventry{Aug 2021 -- May 2023}{M.Sc. in Computational Math Science and Engineering (CMSE)}{Michigan State University}{East Lansing, MI}{\textit{GPA - 3.94}}{} 
\begin{itemize}
    \item \cvitem {Coursework}{Mathematical Foundation for Data Science, Numerical Linear Algebra, Data Mining, Statistical Genetics, Genomic Data Handling: Unix and Python, Genomics and Sequencing Analysis, RNA-Seq Data Analysis}
\end{itemize}
\cventry{Aug 2017 -- May 2021}{B.Sc. in Biosystems Engineering | Minor in CMSE}{Michigan State University}{East Lansing, MI}{\textit{GPA - 3.92}}{} 
\begin{itemize}
    \item Graduated Magna Cum Laude
    \item MasterCard Foundation Scholar Program Recipient
    \item Dean's Honors List : Fall 2021 -- Spring 2021
    \item MasterCard Foundation (MCF) Scholar Program Recipient, 2017
    \item \cvitem{Coursework}{Bioinformatics and Computational Biology, Methods for Parallel Computing, Medical Microbiology}
\end{itemize}

%----------------------------------------------------------------------------------------
%   Research experience
%----------------------------------------------------------------------------------------

\section{Research experience}

\cventry{Aurora, CO}{\textsc{University of Colorado, Anschutz}}{Informatics Research Professional}{Aug 2023 -- Present}{}
{
    \cvitem{Labs}{\textbf{JRaviLab}, \small{Department of Biomedical Informatics | CU Anschutz}, \textbf{Wale Lab}, \small{Microbiology and Molecular Genetics and Integrative Biology | MSU}}
    \cvitem{Mentors}{Dr. Janani Ravi, Dr. Nina Wale}
}
{
     \cvitem{Project 1}{Evolution of bacterial traits in relation to its host association and pathogenicity}
    \begin{itemize} 
        \item Optimized phylogenetic regression analysis with modular R code, speeding up 200+ model iterations and improving insights into bacterial traits and pathogenicity.
        \item Applied Akaike Information Criterion (AIC) for model selection improving accuracy, and realiability of evolutionary biology research.
        %\item Bootstrapped the modelling process to modelling results.
    \end{itemize}
}
{
    \cvitem{Project 2}{MicroGenomeR an R data package for aggregating microbial phenotypic, and genotypic trait.}
    \begin{itemize} 
            \item Adapted Austraits multiple datasets integration pipeline to effectively harmonize 26+ microbial datasets into a central source with consistent units and trait values. 
            \item Enabled efficient storing, and loading of microbial traits data at strain and species levels using memory-optimized file format: parquet and RDA.
    \end{itemize}

    % \cvitem{}{\textbf{Other Projects}}
    % \begin{itemize}
    %     \item Contributing to bugphyzz project.
    %     \item Help in grant writing for creating a biomedical Knowledge graph
    % \end{itemize}
}

\cventry{East Lansing, MI}{\textsc{Michigan State University}}{Technical Aide}{Jun -- Aug 2021}{}
{
    \cvitem{Labs}{\textbf{Juan Steibel Lab}, \small{Department of Animal Science}, MSU}
    \cvitem{Mentor}{Dr. Juan Steibel}
    \cvitem{Project}{Hyperparameter tuning for Long short-term memory (LSTM), model trained on Detecting Agonistic Behavior of Pigs in a Single-Space}
    \begin{itemize}
        \item Leveraged Slurm workload manager for running LSTM deep learning algorithms on high-performance computing clusters, achieving optimized resource utilization and computational efficiency. 
        \item Performed hyper-parameters tunning for LSTM model enhancing model performance on classifying agonistic pig's behaviors.
    \end{itemize}
}

\cventry{East Lansing, MI}{\textsc{Fraunhofer USA Inc.}}{Biosensor Intern}{Jun 2019 -- Aug 2020}{}
{
    \cvitem{Mentor}{Dr. Suzanne Witt}
    \cvitem{Project}{Immobilizing antibodies into Boron-doped diamond surface for detecting COVID-19 spike protein}
    \begin{itemize}
        \item Designed and implemented a data visualization dashboard with Tkinter and Pandas, streamlining the analysis of data from 37 biosensor fabrication experiments and enhancing decision-making.
        \item Functionalized antibody biosensors on boron-doped diamond surfaces using N-hydroxysuccinimide(NHS) techniques, and tested their ability to detect COVID-19 spike protein through electrochemical impedance measurements.
        % \item Involved in testing immobilized antibody biosensors using electrochemical impedance to ascertain the binding of COVID-19 spike protein.
    \end{itemize}
}

\cventry{East Lansing, MI}{\textsc{Michigan State University}}{Undergraduate Research Assistant}{Jun 2019 -- Aug 2020}{}
{
    \cvitem{Labs}{\textbf{MIDI Lab}, \small{Department of Biomedical Engineering}, MSU}
    \cvitem{Mentor}{Dr. Adam Alessio}
    \cvitem{Project}{Classification of ovarion torsion using machine learning with radiological features}
    \begin{itemize}
        \item Performed ovarion torsion classification with radiological features using decision trees classifiers, random trees, and logistic regression through Sci-kit learn
        \item Employed ROC curve, accuracy, specificity, and sensitivity metrics to evaluate models performance on ovarion torsion.
    \end{itemize}
}

%----------------------------------------------------------------------------------------
%  Industry experience
%----------------------------------------------------------------------------------------

\section{Industry experience}
\cventry{Seattle, WA}{\textsc{Amazon}}{Software Development Intern}{May -- Aug 2022}{}
{
    \cvitem{Project}{Developing a central messaging single page application, consolidating messages from different pages}
    \begin{itemize}
        \item Collaborated with teams to design a strategic integration plan, built a React/TypeScript single-page app to improve user experience.
    \end{itemize}
}

%----------------------------------------------------------------------------------------
%   SKILL Section
%----------------------------------------------------------------------------------------
\section{Skills}

\cvitem{Programming languages}{Python, R, Matlab, C++}
\cvitem{Computational tools}{Unix/Linux, Git, High-performance computing clusters}
\cvitem{Methodologies}{Dimensional reduction - Principal Component Analysis(PCA), data analysis and visualization | tidyvverse, ggplot, pandas}
\cvitem{Machine learning}{Linear and Logistic regression, Support Vectors, K-means}
\cvitem{Deep learning}{Neural networks, Transformers | BERT, BioBERT}
\cvitem{NLP}{Term Frequency Inverse Document Frequency (TF-IDF), Text-embedding, LLM}


%--------------------------------------------------------------------------------------
%   Publication
%--------------------------------------------------------------------------------------

\section{Publications}

\begin{enumerate}
    \item Nina Wale*, \textbf{Raymond Lesiyon}*, Clay Cressler, Janani Ravi. \textit{Are bacterial (pathogens) special?}. Manuscript in preparation
    \item \textbf{Raymond Lesiyon}, Janani Ravi. \textit{MicroGenomeR: An R data package for integrating and harmonizing microbial data from various data sources.} Manuscript in preparation
    \item Junjie Jan, Janice Siegford, Dirk Colbry, \textbf{Raymond Lesiyon}, Anna Bosgraaf, Chen Chen, Tomas Norton, Juan Steibel. \textit{Evaluation of Computer Vision for Detecting Agonistic Behaviour of pigs in a Single-SpaceFeeding Stall Through Blocked Cross-Validation Strategies.} 10.2139/ssrn.4098711
    \item Suzanne T Witt, Alexis Rogien, Diana Weiner, James R Siegenthaler, \textbf{Raymond Lesiyon}, Noelle Kurien, Robert Rechenberg, Nina Baule, Aaron Hardy, Michael Becker. \textit{Boron doped diamond thin films for the electrochemical detection of SARS-CoV-2 S1 Protein.} 10.1016/j.diamond.2021.108542
\end{enumerate}

%--------------------------------------------------------------------------------------
%   Presentation
%--------------------------------------------------------------------------------------

\section{Presentation}

\subsection{Research and Technical Talks} 
{
    \begin{itemize}
        \item {\cvitem{July 24}{\textbf{Bioc2024 International Conference}, MicroGenomeR an R data package for aggregating microbial phenotypic, and genotypic trait. Val Andel Institute, Grand Rapids Michigan} }
    \end{itemize}
}

\subsection{Posters} 
{
    \begin{itemize}
        \item {\cvitem{Aug 24}{\textbf{CU Anschutz Department of Biomedical Informatics Retreat}, Are (bacterial) pathogens special? Aurora, CO} }
        \item {\cvitem{June 24}{\textbf{Quantitative Cell and Molecular Biology Symposium}, Are (bacterial) pathogens special? Colorado State University, Fort Collins, CO} }
        \item {\cvitem{April 24}{\textbf{American Society of Microbiology Rocky Mountain Branch}, Are (bacterial) pathogens special? University of Colorado Boulder, Boulder, CO} }
    \end{itemize}
}

%----------------------------------------------------------------------------------------
%   Teaching experience
%----------------------------------------------------------------------------------------

\section{Teaching experience}
{
    \cventry{East Lansing, MI}{\textsc{Michigan State University}}{Graduate Teaching Assistant}{Aug 2021 -- May 2023 }{Dept. of Computational Math Science and Eng.}
    {
        \cvitem{CMSE 802}{Methods in Computational Modeling}
        \cvitem{CMSE 202}{Computational Modelling and Data Analysis II}
    }
    
    \cventry{East Lansing, MI}{\textsc{Michigan State University}}{Academic Tutor}{Jan 2019 -- Aug 2021}{College of Engineering}
    {
        \begin{itemize}
            \item \cvitem{}{Tutored STEM students in calculus and physics, providing guidance and support in their coursework}
        \end{itemize}
    }
}

%----------------------------------------------------------------------------------------
%  Leadership
%----------------------------------------------------------------------------------------

\section{Service and Leadership}
{
    \cventry{Baringo, Kenya}{\textsc{Kokwa Island Primary School}}{Educational Support Initiative}{Aug 2023}{}
    {
        \begin{itemize}
            \item \cvitem{}{Led a school uniform donation initiative to provide uniform to 11 students from orphaned or low-income families, enhancing their access to education. }
        \end{itemize}
    }
    
    \cventry{East Lansing, MI}{\textsc{Michigan State University}}{Academic Core Lead Tutor}{Aug 2020 -- May 2021}{Dept. of Computational Math Science and Eng.}
    {
        \begin{itemize}
            \item \cvitem{}{Guided tutors for smooth tutoring center operations and organized STEM review sessions to prepare students for exams.}
            % \item \cvitem{}{Planned and organized review section for STEM courses, ensuring students were well-prepared for exams}
        \end{itemize}
    }
    
    \cventry{East Lansing, MI}{\textsc{Michigan State University}}{International Orientation Leader }{Aug 2018}{Office of International Students}
    {
        \begin{itemize}
            \item \cvitem{}{Guided a group of 10 international students throughout their first week of enrolling at MSU}
        \end{itemize}
    }
    
    
    \cventry{Bomet, Kenya}{\textsc{Kapkures community}}{Maji Safi ni Uhai Initiative}{Oct 2019}{}
    {
        \begin{itemize}
            \item \cvitem{}{Co-founderd Maji Safi ni Uhai Initiative, and secured \$4000 from MCF program to drill water kapkures community in Bomet, Kenya}
        \end{itemize}
    }

    \cventry{Baringo, Kenya}{\textsc{Baringo South}}{Community Need Assessment Leader}{Mar -- May 2017}{}
    {
        \begin{itemize}
            \item \cvitem{}{Lead a group of 10 students from Education and Social Empowerment Program(EaSEP) conduct community need assessment in Ilchamus community}
        \end{itemize}
    }

    \cventry{Baringo, Kenya}{\textsc{Kokwa Primary School}}{Volunteer Teacher}{Jan -- June 2016}{}
    {
        \begin{itemize}
            \item \cvitem{}{Volunteered to teach STEM classes to grade eight students}
        \end{itemize}
    }
    
    \cventry{Baringo, Kenya}{\textsc{Kokwa Island}}{Community Need Assessment Guide}{Jan 2016}{}
    {
        \begin{itemize}
            \item \cvitem{}{Assisted in community need assesment in Kokwa Island with Friends of Kenya Schools and Wildlife (FKSW) by activating as a translator}
        \end{itemize}
    }
    
}

% %----------------------------------------------------------------------------------------
% %  Awards
% %--------------------------------------------------------------------------------------
% \section{Awards}{
%     \begin{itemize}
%         \item 
%         \item MasterCard Foundation (MCF) Scholar Program Recipient, 2017
%     \end{itemize}
% }

\end{document}
